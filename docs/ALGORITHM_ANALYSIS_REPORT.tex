\documentclass[11pt,a4paper]{article}

% Only core packages
\usepackage[utf8]{inputenc}
\usepackage[T1]{fontenc}
\usepackage{amsmath,amssymb}
\usepackage{graphicx}
\usepackage{booktabs}
\usepackage{hyperref}
\usepackage{geometry}
\usepackage{enumitem}

\geometry{margin=2.5cm}

% Title
\title{%
    \textbf{Critical Analysis of the FPL Team Picker Algorithm} \\[0.5em]
    \large A Technical Review of the Expected Points Model, \\
    Transfer Optimizer, Starting XI Selection, and Captain Strategy
}

\author{Algorithm Analysis Report \\ FPL Team Picker 2025--26}
\date{December 2025}

\begin{document}

\maketitle

\begin{abstract}
This report provides a critical technical analysis of the Fantasy Premier League (FPL) Team Picker algorithm, consisting of four main components: (1) the Machine Learning Expected Points (xP) model with 122-feature engineering, (2) the Linear Programming / Simulated Annealing transfer optimizer, (3) the Starting XI selector with formation enumeration, and (4) the situation-aware captain selection system. We examine the mathematical foundations, implementation choices, and identify both strengths and weaknesses with recommendations for improvement.
\end{abstract}

\tableofcontents
\newpage

%==============================================================================
\section{System Architecture Overview}
%==============================================================================

The FPL Team Picker implements a layered optimization architecture with four distinct layers:

\begin{enumerate}
    \item \textbf{Prediction Layer}: ML Expected Points Service (122 features)
    \item \textbf{Tactical Layer}: Transfer \& Squad Optimization (LP/SA)
    \item \textbf{Selection Layer}: Starting XI \& Captain Selection
    \item \textbf{Risk Layer}: Uncertainty Quantification \& Template Protection
\end{enumerate}

The core decision pipeline is:
\[
\text{Raw Data} \xrightarrow{\text{Feature Engineering}} \text{xP Predictions} \xrightarrow{\text{LP/SA}} \text{Optimal Squad} \xrightarrow{\text{Selection}} \text{Starting XI + Captain}
\]

%==============================================================================
\section{Expected Points Model Training}
%==============================================================================

\subsection{Mathematical Formulation}

The ML Expected Points model predicts the total points $y_i$ for player $i$ in gameweek $t+1$ using:

\begin{equation}
\hat{y}_{i,t+1} = f_\theta(\mathbf{x}_{i,t})
\end{equation}

where $f_\theta$ is a trained ensemble model (RandomForest, XGBoost, LightGBM, or GradientBoosting) and $\mathbf{x}_{i,t} \in \mathbb{R}^{122}$ is the feature vector constructed from historical data up to gameweek $t$.

\subsection{Feature Engineering Architecture (122 Features)}

The \texttt{FPLFeatureEngineer} transformer generates features across 12 categories:

\begin{table}[h]
\centering
\small
\begin{tabular}{llr}
\toprule
\textbf{Category} & \textbf{Description} & \textbf{Count} \\
\midrule
Static Features & Price, position encoding, games played & 4 \\
Cumulative Season Stats & Goals, assists, clean sheets, etc. & 11 \\
Cumulative Per-90 Rates & Goals/90, points/90, xG/90 & 7 \\
Rolling 5GW Form & Recent performance windows & 13 \\
Rolling 5GW Per-90 & Efficiency over recent form & 3 \\
Defensive Metrics & Goals conceded, saves, xGC & 4 \\
Consistency \& Volatility & Std dev of points/minutes, form trend & 3 \\
Team Features & Team-level rolling stats & 13 \\
Fixture Features & Home/away, opponent strength & 6 \\
Enhanced Ownership & Transfer momentum, bandwagon score & 5 \\
Enhanced Value & Points per pound, price volatility & 4 \\
Penalty/Set-Piece & Penalty, corner, FK taker flags & 4 \\
Betting Odds & Win probability, expected goals & 15 \\
Injury/Rotation Risk & Injury risk, rotation likelihood & 5 \\
Venue-Specific Strength & Home/away attack/defense ratings & 6 \\
Player Rankings & Form rank, ICT rank, tackles & 7 \\
Data Quality Indicators & Flags for available data & 5 \\
Elite Interactions & Elite player $\times$ fixture difficulty & 4 \\
\midrule
\textbf{Total} & & \textbf{122} \\
\bottomrule
\end{tabular}
\caption{Feature categories in the ML pipeline}
\end{table}

\subsection{Temporal Leak Prevention}

A critical design principle is \textbf{leak-free features}. All temporal features use strict lookback:

\begin{equation}
\text{rolling\_5gw\_points}_{i,t} = \frac{1}{5} \sum_{k=1}^{5} y_{i,t-k}
\end{equation}

The implementation uses \texttt{shift(1)} operations to ensure no future data leakage.

\textbf{Exception}: Betting odds features are \textit{forward-looking} by nature (available before match kickoff) and require no temporal shift.

\subsection{Uncertainty Quantification}

The system extracts prediction uncertainty from ensemble disagreement:

\begin{equation}
\sigma_i = \text{std}\left(\{f_k(\mathbf{x}_i)\}_{k=1}^{K}\right)
\end{equation}

where $f_k$ represents individual trees in the ensemble. For boosting algorithms (XGBoost, LightGBM), tree contributions are scaled by learning rate:

\begin{equation}
\sigma_i^{\text{XGB}} = \frac{\text{std}(\text{tree contributions})}{\eta}
\end{equation}

where $\eta$ is the learning rate.

\subsection{Strengths of the XP Model}

\begin{itemize}[leftmargin=*]
    \item \textbf{Comprehensive feature set}: 122 features capture diverse signal sources (form, team strength, betting markets, ownership dynamics)
    \item \textbf{Leak-free design}: Strict temporal ordering prevents training-test contamination
    \item \textbf{Uncertainty quantification}: Tree-level variance enables risk-aware decisions
    \item \textbf{Domain-aware imputation}: Position-specific defaults instead of generic fillna(0)
    \item \textbf{Per-gameweek team strength}: Avoids using future information for fixture difficulty
    \item \textbf{Multi-horizon prediction}: Cascading 3GW/5GW predictions use synthetic data
\end{itemize}

\subsection{Weaknesses of the XP Model}

\begin{itemize}[leftmargin=*]
    \item \textbf{Feature redundancy}: Several highly correlated features exist (e.g., \texttt{net\_transfers\_gw} and \texttt{ownership\_velocity}). Feature selection via RFE helps but adds complexity.

    \item \textbf{Limited temporal dynamics}: Rolling 5GW window is fixed; adaptive windows based on recent performance volatility could improve signal extraction.

    \item \textbf{Position-agnostic model}: A single model predicts all positions, yet scoring mechanisms differ significantly (GKP: saves, DEF: clean sheets, FWD: goals). Position-specific models could improve accuracy.

    \item \textbf{No explicit fixture sequence modeling}: Features capture opponent strength but not fixture \textit{patterns} (e.g., tough-easy-tough sequences that affect rotation).

    \item \textbf{Betting odds availability}: Forward-looking betting features assume pre-match odds exist; for future gameweeks, these must be estimated or set to neutral defaults.

    \item \textbf{Cold-start problem}: New players lack historical features. Current solution: fallback to position-based defaults, but this loses player-specific signal.
\end{itemize}

%==============================================================================
\section{Transfer Optimization}
%==============================================================================

\subsection{Problem Formulation}

The transfer optimization is formulated as an Integer Linear Program (ILP):

\begin{align}
\max_{\mathbf{x}} \quad & \sum_{i=1}^{N} \text{xP}_i \cdot x_i - c \cdot \max(0, T - F) \\
\text{s.t.} \quad & \sum_{i=1}^{N} p_i \cdot x_i \leq B \quad \text{(budget)} \\
& \sum_{i \in \mathcal{P}_j} x_i = n_j \quad \forall j \in \{\text{GKP, DEF, MID, FWD}\} \quad \text{(positions)} \\
& \sum_{i \in \mathcal{T}_k} x_i \leq 3 \quad \forall k \in \text{Teams} \quad \text{(team limit)} \\
& \sum_{i \in \mathcal{S}} x_i \geq 15 - M \quad \text{(transfer limit)} \\
& x_i \in \{0, 1\} \quad \forall i
\end{align}

where:
\begin{itemize}
    \item $x_i = 1$ if player $i$ is in the squad
    \item $\text{xP}_i$ is expected points for player $i$
    \item $p_i$ is player price, $B$ is total budget
    \item $c = 4$ is the transfer cost, $F$ is free transfers, $T$ is total transfers
    \item $\mathcal{S}$ is the current squad, $M$ is max transfers allowed
\end{itemize}

\subsection{LP vs SA Comparison}

\begin{table}[h]
\centering
\begin{tabular}{lcc}
\toprule
\textbf{Aspect} & \textbf{Linear Programming} & \textbf{Simulated Annealing} \\
\midrule
Optimality & Guaranteed optimal & Approximate (95-99\%) \\
Speed (transfers) & 1-2 seconds & 10-15 seconds \\
Speed (wildcard) & 1-2 seconds & 30-60 seconds \\
Determinism & Yes & No (seed required) \\
Constraint handling & Native & Penalty functions \\
Non-linear objectives & No & Yes \\
\bottomrule
\end{tabular}
\caption{Comparison of optimization methods}
\end{table}

\subsection{Starting XI Optimization}

For 1-gameweek optimization, the LP additionally optimizes the starting lineup:

\begin{align}
\max \quad & \sum_{i=1}^{N} \text{xP}_i \cdot s_i - \beta \sum_{i=1}^{N} p_i (x_i - s_i) \\
\text{s.t.} \quad & s_i \leq x_i \quad \forall i \quad \text{(can only start if in squad)} \\
& \sum_{i=1}^{N} s_i = 11 \quad \text{(exactly 11 starters)} \\
& \text{Formation constraints: } 1 \text{ GKP}, 3\text{--}5 \text{ DEF}, 2\text{--}5 \text{ MID}, 1\text{--}3 \text{ FWD}
\end{align}

The term $-\beta \sum p_i(x_i - s_i)$ penalizes expensive bench players, encouraging budget allocation to starters.

\subsection{Strengths of Transfer Optimization}

\begin{itemize}[leftmargin=*]
    \item \textbf{Guaranteed optimality}: LP provides provably optimal solutions for the defined objective

    \item \textbf{10-50x speedup}: CBC solver finds solutions in 1-2 seconds vs. 30+ seconds for SA

    \item \textbf{Joint squad-and-XI optimization}: For 1GW, simultaneously optimizes who to buy and who to start

    \item \textbf{Proper budget handling}: Uses selling price for current players, market price for new

    \item \textbf{Safety check}: If LP solution is worse than keeping current squad, reverts to ``no transfer''

    \item \textbf{Forced transfer handling}: Correctly handles unavailable players (injured/suspended)

    \item \textbf{Free Hit support}: Recognizes Free Hit chip and optimizes for 1GW only
\end{itemize}

\subsection{Weaknesses of Transfer Optimization}

\begin{itemize}[leftmargin=*]
    \item \textbf{Myopic objective}: LP maximizes xP for current horizon but doesn't consider:
    \begin{itemize}
        \item Future fixture swings (selling Salah before DGW)
        \item Price rises/falls and team value growth
        \item Upcoming BGW/DGW chip strategies
    \end{itemize}

    \item \textbf{Linear objectives only}: Cannot directly optimize non-linear objectives like Sharpe ratio or variance minimization.

    \item \textbf{No exploration}: LP always returns the same optimal point. For wildcards, users might want multiple near-optimal alternatives.

    \item \textbf{Price volatility ignored}: A player about to rise \pounds 0.2m might be more valuable than xP alone suggests.

    \item \textbf{Rolling transfer banking not modeled}: Cannot model ``bank this week to have 2 FT next week''.
\end{itemize}

%==============================================================================
\section{Starting XI Selection}
%==============================================================================

\subsection{Formation Enumeration}

The system enumerates all valid FPL formations and selects the one maximizing starting XI xP:

\begin{table}[h]
\centering
\begin{tabular}{ccccc}
\toprule
Formation & GKP & DEF & MID & FWD \\
\midrule
3-4-3 & 1 & 3 & 4 & 3 \\
3-5-2 & 1 & 3 & 5 & 2 \\
4-3-3 & 1 & 4 & 3 & 3 \\
4-4-2 & 1 & 4 & 4 & 2 \\
4-5-1 & 1 & 4 & 5 & 1 \\
5-3-2 & 1 & 5 & 3 & 2 \\
5-4-1 & 1 & 5 & 4 & 1 \\
\bottomrule
\end{tabular}
\caption{Valid FPL formations}
\end{table}

\textbf{Algorithm}:
\begin{enumerate}
    \item Sort players by xP within each position
    \item For each valid formation $(d, m, w)$:
    \begin{enumerate}
        \item Select top 1 GKP, top $d$ DEF, top $m$ MID, top $w$ FWD
        \item Calculate total xP for this formation
    \end{enumerate}
    \item Return formation with maximum xP
\end{enumerate}

\subsection{Availability Filtering}

Players with status $\in \{i, s, u\}$ (injured, suspended, unavailable) are excluded from starting XI consideration \textit{before} formation enumeration.

\subsection{Strengths of Starting XI Selection}

\begin{itemize}[leftmargin=*]
    \item \textbf{Exhaustive search}: All 7 valid formations evaluated, guaranteeing optimal selection

    \item \textbf{xP-adaptive}: Naturally adapts formation based on where xP is concentrated

    \item \textbf{Availability-aware}: Automatically excludes unavailable players

    \item \textbf{Multi-horizon support}: Can optimize for 1GW, 3GW, or 5GW xP
\end{itemize}

\subsection{Weaknesses of Starting XI Selection}

\begin{itemize}[leftmargin=*]
    \item \textbf{No auto-sub modeling}: FPL auto-substitutes bench players if starters don't play. Optimal bench order should consider auto-sub probabilities:
    \[
    \mathbb{E}[\text{Points}] = \sum_{i \in \text{XI}} \text{xP}_i \cdot P(\text{plays}_i) + \sum_{j \in \text{bench}} \text{xP}_j \cdot P(\text{autosub}_j)
    \]

    \item \textbf{Binary availability}: Players are either available or excluded. Doesn't model partial availability (e.g., 75\% chance of playing).

    \item \textbf{Formation fixation}: Once selected, formation is fixed. No consideration of differential formations.

    \item \textbf{No doubtful player strategy}: A doubtful premium might be better benched for a certain budget player to avoid 1-point cameos.
\end{itemize}

%==============================================================================
\section{Captain Selection}
%==============================================================================

\subsection{Upside-Seeking Captain Score}

The captain selection uses a \textbf{ceiling-seeking} approach rather than expected value:

\begin{equation}
\text{Captain Score}_i = 2 \cdot \text{xP}_i^{(90)} \cdot (1 + \alpha_{\text{own}} + \alpha_{\text{match}})
\end{equation}

where the 90th percentile upside is:
\begin{equation}
\text{xP}_i^{(90)} = \text{xP}_i + 1.28 \cdot \sigma_i
\end{equation}

\subsection{Template Protection}

For high-ownership players ($>50\%$ ownership), a template protection bonus applies:

\begin{equation}
\alpha_{\text{own}} = \min\left(\frac{\text{ownership} - 50}{20}, 1\right) \cdot \text{fixture\_quality} \cdot 0.40
\end{equation}

This ensures that highly-owned players in favorable fixtures receive up to 40\% scoring boost, preventing rank-damaging differential picks against template.

\subsection{Situation-Aware Strategy}

The intelligent captain recommendation adapts strategy based on manager context:

\begin{table}[h]
\centering
\small
\begin{tabular}{lll}
\toprule
\textbf{Rank Category} & \textbf{Season Phase} & \textbf{Recommended Strategy} \\
\midrule
Elite ($<100k$) & Late & Template Lock \\
Elite ($<100k$) & Early/Mid & Protect Rank \\
Comfortable ($<500k$) & Any & Balanced \\
Chasing ($<2M$) & Declining momentum & Chase Rank \\
Trailing ($>2M$) & Any & Maximum Upside \\
Any & Triple Captain active & Maximum Upside \\
Any & Free Hit active & Chase Rank \\
\bottomrule
\end{tabular}
\caption{Captain strategy auto-detection logic}
\end{table}

\subsection{Haul Probability Matrix}

The system calculates outcome probabilities assuming normal distribution:

\begin{align}
P(\text{blank}) &= \Phi\left(\frac{\tau_{\text{blank}} - \text{xP}}{\sigma}\right) \\
P(\text{return}) &= \Phi\left(\frac{\tau_{\text{return}} - \text{xP}}{\sigma}\right) - P(\text{blank}) \\
P(\text{haul}) &= 1 - \Phi\left(\frac{\tau_{\text{return}} - \text{xP}}{\sigma}\right)
\end{align}

where $\tau_{\text{blank}} = 2$ and $\tau_{\text{return}} = 8$ are configurable thresholds.

\subsection{Strengths of Captain Selection}

\begin{itemize}[leftmargin=*]
    \item \textbf{Ceiling-seeking philosophy}: Using 90th percentile prioritizes haul potential---correct for doubling points

    \item \textbf{Template protection}: Avoids rank-destroying differential picks when template captain is clearly optimal

    \item \textbf{Situation awareness}: Auto-detects appropriate strategy based on rank and season phase

    \item \textbf{Betting odds integration}: Uses team win probability and expected goals for matchup quality

    \item \textbf{Haul probability transparency}: Shows blank/return/haul probabilities to inform decision

    \item \textbf{Rank impact estimation}: Models differential potential based on ownership
\end{itemize}

\subsection{Weaknesses of Captain Selection}

\begin{itemize}[leftmargin=*]
    \item \textbf{Gaussian assumption}: Real FPL point distributions are heavily skewed (many 2s, occasional 15+). Normal distribution underestimates blank probability.

    \item \textbf{No historical captain accuracy tracking}: System doesn't learn from past captain decisions. A feedback loop could improve calibration.

    \item \textbf{Vice-captain underutilized}: Simply second-highest scorer. Should consider correlation with captain and ownership differential.

    \item \textbf{Simplistic rank impact model}: Uses rough heuristics. Could be improved with historical rank movement data.

    \item \textbf{No mini-league awareness}: Captain strategy should differ in head-to-head mini-leagues (target opponent's captain).

    \item \textbf{Double gameweek captain}: No special handling for players with two fixtures---should boost their ceiling.
\end{itemize}

%==============================================================================
\section{End-to-End System Analysis}
%==============================================================================

\subsection{Data Pipeline}

The data flow through the system:

\begin{enumerate}
    \item \textbf{Data Loading}: 12 data sources including historical performance, fixtures, betting features

    \item \textbf{Feature Engineering}: Raw data $\rightarrow$ 122 leak-free features

    \item \textbf{Model Training}: Walk-forward CV with temporal splits

    \item \textbf{Inference}: Load trained model, generate xP predictions with uncertainty

    \item \textbf{Optimization}: LP/SA for transfer decisions

    \item \textbf{Selection}: Starting XI and bench order

    \item \textbf{Captain}: Situation-aware captain pick
\end{enumerate}

\subsection{Cross-Validation Strategy}

Walk-forward validation ensures temporal integrity. For training on GW 1--$n$, test on GW $n+1$. The first trainable GW is 6 due to 5GW rolling feature requirements.

\subsection{Model Comparison Framework}

The system supports systematic model comparison:

\begin{equation}
\text{Composite Score} = 0.4 \cdot \text{MAE} + 0.3 \cdot \text{Spearman} + 0.2 \cdot \text{RMSE} + 0.1 \cdot \text{Captain Accuracy}
\end{equation}

%==============================================================================
\section{Identified Gaps and Recommendations}
%==============================================================================

\subsection{Strategic Gaps}

\begin{enumerate}
    \item \textbf{BGW/DGW Planning}: System lacks proactive handling of blank and double gameweeks. \textit{Recommendation}: Integrate FPL calendar with lookahead optimization.

    \item \textbf{Chip Strategy}: No systematic chip timing optimization. \textit{Recommendation}: Add chip assessment with simulation.

    \item \textbf{Team Value Management}: Price change prediction not integrated. \textit{Recommendation}: Add expected value growth as secondary objective.

    \item \textbf{Transfer Banking}: LP cannot model ``save transfer'' strategy. \textit{Recommendation}: Multi-stage stochastic programming or heuristic rules.
\end{enumerate}

\subsection{Model Improvements}

\begin{enumerate}
    \item \textbf{Position-Specific Models}: Train separate models for each position to capture scoring differences.

    \item \textbf{Non-Gaussian Uncertainty}: Replace normal assumption with empirical distribution for more accurate haul/blank probabilities.

    \item \textbf{Adaptive Rolling Windows}: Use player-specific window sizes based on form stability.

    \item \textbf{Auto-Sub Modeling}: Incorporate playing time uncertainty into starting XI decisions.
\end{enumerate}

\subsection{Optimization Improvements}

\begin{enumerate}
    \item \textbf{Multi-Week LP}: Extend LP to 3-5 gameweek horizon with transfer continuity constraints.

    \item \textbf{Robust Optimization}: Optimize for worst-case or CVaR under prediction uncertainty.

    \item \textbf{Exploration Mode}: SA generates diverse near-optimal squads for wildcard exploration.
\end{enumerate}

%==============================================================================
\section{Conclusion}
%==============================================================================

The FPL Team Picker implements a sophisticated multi-stage optimization system with notable strengths:

\begin{itemize}
    \item \textbf{Comprehensive ML pipeline}: 122 leak-free features with uncertainty quantification
    \item \textbf{Provably optimal transfers}: LP guarantees optimal solution for defined objective
    \item \textbf{Situation-aware captaincy}: Adapts strategy to rank, season phase, and chip status
    \item \textbf{Clean architecture}: Well-separated domain services with clear responsibilities
\end{itemize}

Key areas for improvement:

\begin{itemize}
    \item \textbf{Strategic planning}: BGW/DGW awareness, chip optimization, team value growth
    \item \textbf{Distributional modeling}: Non-Gaussian uncertainty for more accurate captain analysis
    \item \textbf{Multi-week horizon}: Extend LP to consider future fixture swings
    \item \textbf{Position-specific models}: Capture scoring mechanism differences across positions
\end{itemize}

The system provides a strong foundation for competitive FPL management, with the LP optimizer representing a significant improvement over pure heuristic approaches. The main limitation is the myopic nature of single-gameweek optimization---extending to multi-week planning would provide substantial strategic benefit.

\appendix

\section{Mathematical Notation Summary}

\begin{table}[h]
\centering
\begin{tabular}{ll}
\toprule
\textbf{Symbol} & \textbf{Description} \\
\midrule
$y_i$ & Actual points for player $i$ \\
$\hat{y}_i$ & Predicted expected points (xP) \\
$\sigma_i$ & Prediction uncertainty (std dev) \\
$x_i$ & Binary decision variable (1 if in squad) \\
$s_i$ & Binary starter variable (1 if in starting XI) \\
$p_i$ & Player price \\
$B$ & Total budget \\
$F$ & Free transfers \\
$c$ & Transfer cost (4 points) \\
$\mathcal{P}_j$ & Set of players in position $j$ \\
$\mathcal{T}_k$ & Set of players from team $k$ \\
$\mathcal{S}$ & Current squad player set \\
\bottomrule
\end{tabular}
\end{table}

\section{Configuration Parameters}

Key configurable parameters in the system:

\begin{verbatim}
# Feature Engineering
rolling_window: 5  # GW lookback for form features

# LP Optimization
transfer_cost: 4  # Points penalty per transfer
max_transfers: 5  # Maximum transfers allowed

# Captain Selection
elite_rank_threshold: 100000
comfortable_rank_threshold: 500000
chasing_rank_threshold: 2000000
blank_threshold: 2  # Points below this = blank
return_threshold: 8  # Points above this = haul
template_ownership_threshold: 50  # % for template
\end{verbatim}

\end{document}
